%-------------------------------------------------------------------------------
%	SECTION TITLE
%-------------------------------------------------------------------------------
\cvsection{Projects}


%-------------------------------------------------------------------------------
%	CONTENT
%-------------------------------------------------------------------------------
\begin{cventries}
\cventry
    {\textnormal{InceptionV3, LSTM, Angular, Typescript, JQuery, Bootstrap, HTML5, CSS, Flask}} % Role
    {Captionify.me | \href{https://github.com/Rpmalukani22/Automatic-Image-Caption-Generator}{ Link}} % Event
    { } % Location
    {Apr 2020} % Date(s)
    {
      \begin{cvitems} % Description(s)
        \item {A Deep Learning based web-application to generate captions for a given image. Convolutional Neural Network (InceptionV3 with transfer learning approach) was used for image feature extraction. Long Short Term Memory (LSTM) was used to handle linguistic aspects.}
        % \item {}
        % \item{}
        \item{Fairly accurate captions were obtained by implementing Merge architecture for Encoder-Decoder Model.}
        \item {Trained model was integrated with web-application which consists of Angular framework-based frontend and Flask framework-based backend.}
        \item{ Client-side Text-to-Speech (TTS) component was responsible for converting the predicted caption to speech.}
        \end{cvitems}
    }
%---------------------------------------------------------
  \cventry
    {\textnormal{Kotlin, SQLite}} % Role
    {My Instant Notes | \href{https://play.google.com/store/apps/details?id=dropdwn.ruchitesh.com.notes}{ Link}} % Event
    { } % Location
    {Dec 2017} % Date(s)
    {
      \begin{cvitems} % Description(s)
        \item {It is an Android application for taking notes and maintaining memos in 10 different languages.}
        \item{Facilities to add, modify, delete, search and save the notes at the user's end were provided using the SQLite database.}
        \item{The application was built using Koltin programming language in Android Studio IDE.}
      \end{cvitems}
      }
  \vspace{0.4 cm}
%---------------------------------------------------------
    \cventry
    {\textnormal{Kotlin}} % Role
    {CalcPro | \href{https://github.com/Rpmalukani22/CALCPRO-app}{ Link}} % Event
    { } % Location
    {Nov 2017} % Date(s)
    {
      \begin{cvitems} % Description(s)
        \item {It is a calculator application for android.}
        \item{The application was built using Koltin programming language in Android Studio IDE.}
        \item{The application supports layouts of different devices of various screen sizes.}
        % \item{The app has obtained 500+ national and international downloads and 4.9/5 ratings on Google PlayStore with zero crash report till date.}
        % % \item {Introduced how to freely host the weIt is an Android application for taking notes and maintaining memos in 10 different languagesb application with high performance utilizing global CDN services.}
      \end{cvitems}
    }

%---------------------------------------------------------
 \cventry
    {\textnormal{HTML, CSS, Javascript, Jquery,Node.js and MongoDB}} % Role
   { Restful blog application | \href{https://github.com/Rpmalukani22/RESTful-Blog-app}{ Link}} % Event
    { } % Location
    {Sept 2018} % Date(s)
    {
      \begin{cvitems} % Description(s)
        \item { RESTful Blog App is a web application developed with best applications of RESTful Routing using Node.JS, Express.JS and Embedded JavaScript (EJS). It allows users to create, edit and delete a post. }
        \item{NoSQL database(MongoDB) was used to provide the database related functionalities.}
        % \item{The app has obtained 500+ national and international downloads and 4.9/5 ratings on Google PlayStore with zero crash report till date.}
        % % \item {Introduced how to freely host the weIt is an Android application for taking notes and maintaining memos in 10 different languagesb application with high performance utilizing global CDN services.}
      \end{cvitems}
    }

%---------------------------------------------------------
    \cventry
    {\textnormal{Node.js}} % Role
    {Weather application | \href{https://github.com/Rpmalukani22/weather-app}{ Link}} % Event
    { } % Location
    {Jul 2018} % Date(s)
    {
      \begin{cvitems} % Description(s)
        \item {
      A Node.js application which shows weather information of given location using Dark Sky API.
 }
        % \item{The application .}
      % \item{NoSQL database(MongoDB) was used to provide the database related functionalities.}
        % \item{The app has obtained 500+ national and international downloads and 4.9/5 ratings on Google PlayStore with zero crash report till date.}
        % % \item {Introduced how to freely host the weIt is an Android application for taking notes and maintaining memos in 10 different languagesb application with high performance utilizing global CDN services.}
      \end{cvitems}
  } 
\end{cventries}
